\chapter{Related Work}
\label{ch:related_work}

\section[INLOOM]{INLOOM}
This work aims to validate the quality of the results the INLOOM Software \cite{1}
produces. For that reason it is inevitable to take a look into how it evaluates
student solutions and how it persists the collected information. INLOOM is an
acronym for \textit{INnteractive Learning center for Object-Oriented Modelling}. 

The software, as the name suggests, is intended to be employed in a learning environment.
It is used to evaluate student solutions to modelling tasks, the students have to work 
on as part of the mandatory beginner software engineering course at TU Dresden.

\section[Quality-Testing in existing ITS]{Quality-Testing in existing ITS}
Due to rising student numbers and the availablity of new interfacing technologies the
interest, in automatic evaluation of student modelling work, has increased in recent 
years. Even though: Functioning evaluation tools and methods remain scarce. 
Starting point for the research into existing ITS, was a collection of such, referenced
in \cite{1}. Since those influenced decision made during the design of INLOOM, it is
only natural to also include them 

% TODO: How did I perform my research?

\section[Supervised Learning]{Supervised Learning}