\chapter{Introduction}
\label{ch:introduction}

% TODO: Explain what INLOOM is and does here, or in related work?

\section[Motivation]{Motivation}

2020 was jinxed. The Covid-19 pandemic changed how our lives work.
It has presented the globalised world with little anticipated challenges
and we can feel its influence almost every aspect of our everyday life. 
In order to reduce the amount of human contacts as much as possible, every 
aspect of human interaction was evaluated for its digitizability. However: 
We were all forced to witness, that our digital infracstructe is most obiously 
not yet up to the task of enabling us to \textit{live the remote life}.

Most of the fundamental problems were trivialities, like missing webcams or 
a too slow internet connection. Where such were taken care of, the harder-to-fix
problems came to light. Problems like inadequately educated overtaxed personnel 
and missing software solutions, that comply with european digital privacy regulation. 
Even though at first glance the personnel problems don't seem to matter much to 
software developers, it are still problems, which can and will, at least in part,
be resolved by them. \\

I don't want to claim, that the digitization of everyday life was a complete
failure though. Like me, most office workers were able to migrate to home office 
without much fuss. Still: Living the life of a remote student for half a year,
definitely motivates me, to spend some thought on how to make e-teaching a 
little better. 

Even though it prooved hazardous sometimes, working with existing e-teaching 
tools made me realize, what huge potential lies within a properly digitized higher
eduaction. Such would not only help temper the effects emergencies, like the covid
pandemic, have on university life, but will also be a powerful tool in futureproofing
universities for the challenges of rising student numbers in the years to come.

Intelligent-Tutoring-Systems like INLOOM, and ITS under active developement at TU 
Dresden will make the increased workload managable for university personnel. 
Developers are required to produce software that is as intuitive as possible, provides
a decent grade of digital security, complies with privacy regulations, handles high 
traffic without complain and all that, while providing an unquestionably accurate and 
fair environment for everyone involved. 

Integrating digital resources into the workflow seamlessly, will enable teaching 
personnel to still be able to focus on the individual student, when the student
groups they teach become way bigger, than they are today - Which, in the end, could be
the only way to keep higher education as we know it affordable. By contributing to 
INLOOM, I contribute to digitizing education.

\section[Research Questions]{Research Questions}

In this thesis I am tasked with finding or developing a concept for validating
evaluations INLOOM generates for student created UML-models, in order to ensure
the correctness and fairness of said evaluations. For that purpose I will aim to
answer the following research questions:

